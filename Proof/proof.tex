\documentclass{article}
\usepackage[english]{babel}
\usepackage{amsthm}
\usepackage{amsmath}

\newtheorem{prop}{Proposition}
\newtheorem{cor}{Corollary}
\begin{document}

\begin{prop}
  As $N \rightarrow \infty$ and $t \rightarrow \infty$, the ratio of $\mathbf{X}_b$ to $\mathbf{X}_i$, $i\ne b$ in solutions to the system with positive coefficients, initial conditions $\mathbf{X} = \mathbf{0}$, and $B_i \ne B_j$ for all $i \ne j$
  \begin{equation}
    \frac{d \mathbf{X}_i}{dt} = (A_i + B_i\mathbf{X}_i)\left(N - \sum_{j=1}^J\mathbf{X}_j\right) - \frac{sd_i\mathbf{X}_i}{K+L_i\mathbf{X}_i}, i \in 1, 2, \dots, J
    \label{theeq}
  \end{equation}
  approaches $\infty$ where $\mathbf{X}_b$ is the component of $\mathbf{X}$ with the largest value of $B_i$ in (\ref{theeq}).
\end{prop}
\begin{proof}
  We begin by noting that for solutions to (\ref{theeq}) starting at $\mathbf{X} = \mathbf{0}$, $\mathbf{X}_i \ge 0$ for all $i$ for all $t$. This is because when $\mathbf{X}_i = 0$, (\ref{theeq}) becomes $\frac{d\mathbf{X}_i}{dt} = A_i$, which is positive. Additionally, we note that for solutions to (\ref{theeq}) starting at $\mathbf{X} = \mathbf{0}$, $\sum_{j=1}^J\mathbf{X}_j < N$ for all $t$. This is because, when $\sum_{j=1}^J\mathbf{X}_j = N$, (\ref{theeq}) becomes 
  \begin{equation}
  	\frac{d\mathbf{X}_i}{dt} = -\frac{sd_i\mathbf{X}_i}{K + L_i\mathbf{X}_i}
	\label{notbig}
  \end{equation}
  which is non-positive for all values of $i$ because all coefficents are positive and $\mathbf{X}_i \ge 0$ by the earlier note. (\ref{notbig}) is negative for at least one component of $\mathbf{X}$ because the only way for it to be zero for all values of $i$ is if $\mathbf{X} = \mathbf{0}$, which is impossible because we set $\sum_{j=1}^J \mathbf{X}_j = N$ and $N > 0$. Thus $\sum_{j=1}^J\mathbf{X}_j < N$ for all $t$.
  
  Next, we see that (\ref{theeq}) is asymptotically equivalent to
  \begin{equation}
    \frac{d\mathbf{X}_i}{dt} = (A_i + B_i \mathbf{X}_i)\left(N-\sum_{j=1}^J \mathbf{X}_j\right)
    \label{theeq2}
  \end{equation}
  as $N \rightarrow \infty$ with initial condition $\mathbf{X} = \mathbf{0}$ because
  \begin{align}
    \lim_{N \rightarrow \infty} \frac{(A_i + B_i\mathbf{X}_i)\left(N - \sum_{j=1}^J\mathbf{X}_j\right) - \frac{sd_i\mathbf{X}_i}{K+L_i\mathbf{X}_i}}{ (A_i + B_i \mathbf{X}_i)\left(N-\sum_{j=1}^J \mathbf{X}_j\right)} &= \\
    \lim_{N\rightarrow\infty} 1 - \frac{sd_i\mathbf{X}_i}{(K+L_i\mathbf{X}_i)(A_i+B_i\mathbf{X}_i)(N - \sum_{j=1}^J\mathbf{X}_j)} &= 1
        \label{theeqratio}
  \end{align}
  and $\sum_{j=1}^J < N$ by the earlier note which implies that $N - \sum_{j=1}^J \mathbf{X}_j \ne 0$. Thus as $N \rightarrow \infty$, the solutions to $(\ref{theeq})$ approach those of $(\ref{theeq2})$. Looking at $(\ref{theeq2})$, we see that the derivative of each component has the common factor $N - \sum_{j=1}^J \mathbf{X}_j$. Therefore, the solution curve to $(\ref{theeq2})$ starting at $\mathbf{X} = \mathbf{0}$ has the same shape as that of
  \begin{equation}
    \frac{d \mathbf{X}_i}{dt} = (A_i + B_i\mathbf{X}_i).
    \label{theeq3}
  \end{equation}
  for $\sum_{j=1}^J \mathbf{X}_j < N$. Using separation by parts, the solution to $(\ref{theeq3})$ with initial condition $\mathbf{X} = \mathbf{0}$ is
  \begin{equation}
    \mathbf{X}_i(t) = \frac{A_i}{B_i}\left(e^{B_it}-1\right)
    \label{xsol}
  \end{equation}
  Denote the component of $\mathbf{X}_i(t)$ with the largest value of $B_i$ as component $b$. We next note that if $B_i > B_j$, the limit of the ratio between any two components $\mathbf{X}_i(t)$ and $\mathbf{X}_j(t)$ as $t \rightarrow \infty$ is
  \begin{equation}
    \lim_{t \rightarrow \infty} \frac{ \frac{A_i}{B_i}\left(e^{B_it}-1\right)} {\frac{A_j}{B_j}\left(e^{B_jt}-1\right)
} = \infty
    \label{ratio}
  \end{equation}
  Therefore, the ratio between $\mathbf{X}_b(t)$ and any other component of $\mathbf{X}(t)$ must approach $\infty$ as $t \rightarrow \infty$. As $N \rightarrow \infty$, the time for which $\sum_{j=1}^J \mathbf{X}_j < N$ also approaches $\infty$. As (\ref{theeq2}) and (\ref{theeq3}) have solutions which follow the same curve while $\sum_{j=1}^J \mathbf{X}_j < N$, this means that the same conclusion about ratios of components holds for solutions to (\ref{theeq2}) as $N \rightarrow \infty$ and $t \rightarrow \infty$.

  Thus, as solutions with the initial condition $\mathbf{X} = \mathbf{0}$ to (\ref{theeq}) approach those of (\ref{theeq2}) which in turn approach those of (\ref{theeq3}) as $N \rightarrow \infty$, the ratio of $\mathbf{X}_b$ in solutions to (\ref{theeq}) to any other component must approach $\infty$ as $N \rightarrow \infty$ and $t \rightarrow \infty$.

\end{proof}

\begin{cor}
  There exist values of $N$ for which the component $\mathbf{X}_b$, which is defined as in the above proof, of the solution to (\ref{theeq}) with initial condition $\mathbf{X} = \mathbf{0}$ is larger than any other component as $t \rightarrow \infty$.
\end{cor}
\begin{proof}
  Assume for contradiction that there were no values of $N$ for which the corollary held true. Then, there would not exist a value of $N$ for which $\mathbf{X}_b > \mathbf{X}_i$ for $i \ne b$ as $t \rightarrow \infty$. Therefore, the ratio between $\mathbf{X}_b$ and any other component could not approach $\infty$ as $N \rightarrow \infty$ and $t \rightarrow \infty$ because there would be no values of $N$ for which $\mathbf{X}_b$ were greater than any other component. This contradicts the above proof. Thus, the corollary holds true.
\end{proof}
\end{document}
